Este informe aborda el problema de la distancia de Damerau-Levenshtein mediante dos enfoques principales: 
algoritmos de fuerza bruta y programación dinámica. Se analiza cómo los costos asociados a operaciones como eliminación,
inserción, reemplazo y transposición afectan la complejidad temporal de las soluciones propuestas. El trabajo incluye un diseño 
detallado de los algoritmos, su implementación en código y la evaluación experimental en diversos conjuntos de datos. Los resultados 
muestran una clara ventaja en eficiencia para la programación dinámica, aunque a costa de un mayor uso de memoria, mientras que la 
fuerza bruta puede ser útil en casos específicos con cadenas vacías.

