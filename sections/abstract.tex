Este informe estudia la distancia de Damerau-Levenshtein utilizando algoritmos de fuerza bruta y programación dinámica,
comparando sus eficiencias mediante experimentos en diferentes escenarios, y el impacto que tiene la busqueda de menores costos de edición 
en el rendimiento. Los objetivos incluyen analizar estas diferencias, optimizar los enfoques y evaluar su aplicabilidad. La 
infraestructura utilizada incluye hardware con procesador AMD Ryzen 5 4600H y herramientas como Python y C++ en un entorno WSL. 
Los resultados confirman la ventaja en eficiencia temporal de la programación dinámica, aunque con mayor uso de memoria, mientras 
que la fuerza bruta es útil en casos simples como cadenas vacías.
