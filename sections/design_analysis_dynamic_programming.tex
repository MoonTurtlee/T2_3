%\epigraph{\textit{Dynamic programming is not about filling in tables. It's about smart recursion!}}{\citeauthor{algorithms_erickson}, \citeyear{algorithms_erickson} \cite{algorithms_erickson}}

\subsubsection{Descripción de la solución recursiva}
    
    Por cada llamada a la función es necesario evaluar los cuatro casos posibles (eliminacion, inserción, reemplazo y transposición) y escoger el de menor costo,
    si guardamos el costo asociado a esa llamada este nos puede servir para evitar realizar el mismo calculo si se da el caso, de esta manera se asegura que cada llamada
    que posea los mismos parametros se resuelva una única vez, esto implica que mediante la consulta recurrente de valores podemos
    disminuir el costo temporal del problema en comparación a la utilizacion de fuerza bruta y obtendremos exactamente la misma respuesta final. 
    

\subsubsection{Relación de recurrencia}
    
    \begin{itemize}
        \item \textit{(condicion)} Si el valor de $Cache[i][j]$ es distinto de $-1$ se retorna el valor de $Cache[i][j]$ 
        \item \textit{(caso base)} Si el valor de $i$ es igual al largo de $S1$ y el valor de $j$ igual al largo de $S2$ entonces se retorna 0 
        \item \textit{(condicion)} Si el valor de $i$ iguala el largo de $S1$ entonces solo se realizan inserciones por cada caracter faltante por cubrir de $S2$, devolviendo el costo de estas operaciones.
        \item \textit{(condicion)} Si el valor de $j$ iguala el largo de $S2$ entonces solo se realizan eliminaciones por cada caracter sobrante de $S1$, devolviendo el costo de estas operaciones.
        \item \textit{(Recurrencia)}
                \[
                DP(S1, S2, i, j) = \text{Cache}[i][j] = min
                \left\{
                \begin{aligned}
                    &\text{costo\_del}(S1[i]) + DP(S1, S2, i+1, j), \\[0.5em]
                    &\text{costo\_ins}(S2[j]) + DP(S1, S2, i, j+1), \\[0.5em]
                    &\text{costo\_sub}(S1[i], S2[j]) + DP(S1, S2, i+1, j+1), \\[0.5em]
                    &\text{costo\_trans}(S1[i], S1[i+1]) + DP(S1, S2, i+2, j+2)
                \end{aligned}
                \right\}
                \]
    \end{itemize}
    

\subsubsection{Identificación de subproblemas}

Los subproblemas son las combinaciones de las posiciones $i$ y $j$
que representan el costo mínimo para transformar la subcadena de $S1$ desde $i$ a
la subcadena de $S2$ desde $j$.

\subsubsection{Estructura de datos y orden de cálculo}

Se utiliza el enfoque top-down una tabla de \texttt{Cache} para almacenar los resultados, que son calculados 
recursivamente, garantizando que cada subproblema solo se evalúe una vez.

\subsubsection{Algoritmo utilizando programación dinámica}

\renewcommand\thealgocf{2}
\begin{algorithm}[H]
    \SetKwProg{myproc}{Procedure}{}{end}
    \SetKwFunction{DP}{DP}
    \KwIn{Cadenas $S1$, $S2$, índices $i$, $j$, y matriz de cache de tamaño (n+1) $\times$ (m+1) inicializada en -1 }
    
    \footnotesize

    \myproc{\DP{S1, S2, i, j, Cache}}{
        \If{$Cache[i][j] \neq -1$}{
            \Return $Cache[i][j]$
        }

        \If{$i = \text{largo}(S1)$ \textbf{and} $j = \text{largo}(S2)$}{
            \Return $Cache[i][j] = 0$
        }
        
        \If{$i = \text{largo}(S1)$}{
            \Return $Cache[i][j] = \text{costo\_ins}(S2[j]) + \text{DP}(S1, S2, i, j+1, Cache)$
        }
        
        \If{$j = \text{largo}(S2)$}{
            \Return $Cache[i][j] = \text{costo\_del}(S1[i]) + \text{DP}(S1, S2, i+1, j, Cache)$
        }
        
        $elim \gets \text{costo\_del}(S1[i]) + \text{DP}(S1, S2, i+1, j, Cache)$\;
        $ins \gets \text{costo\_ins}(S2[j]) + \text{DP}(S1, S2, i, j+1, Cache)$\;
        $remp \gets \text{costo\_sub}(S1[i], S2[j]) + \text{DP}(S1, S2, i+1, j+1, Cache)$\;
        $transp \gets \infty$\;
        
        \If{$i+1 < \text{largo}(S1)$ \textbf{and} $j+1 < \text{largo}(S2)$ \textbf{and} $S1[i] = S2[j+1]$ \textbf{and} $S1[i+1] = S2[j]$}{
            $transp \gets \text{costo\_trans}(S1[i], S1[i+1]) + \text{DP}(S1, S2, i+2, j+2, Cache)$
        }
        
        \Return $Cache[i][j] = \min(elim, ins, remp, transp)$
    }
    \caption{Algoritmo de Programación Dinámica para la Transformación de Cadenas}
    \label{alg:mi_algoritmo_2}
\end{algorithm}


\subsubsection{Ejemplo de ejecución}
Para ilustrar la ejecución del algoritmo de programación dinámica, consideremos las siguientes cadenas de largo 3:  
\[
S_1 = \texttt{abc}, \quad S_2 = \texttt{adc}
\]

Supongamos que los costos de las operaciones son:  
\begin{itemize}
    \item \( \text{costo\_ins}(x) = 1 \)
    \item \( \text{costo\_del}(x) = 1 \)
    \item \( \text{costo\_sub}(x, y) = 1 \) si \( x \neq y \), \( 0 \) en caso contrario
    \item \( \text{costo\_trans}(x, y) = 1 \)
\end{itemize}

Inicializamos una matriz de caché de dimensiones \( 4 \times 4 \) (longitudes de \( S_1 \) y \( S_2 \) más 1), con todos sus valores en \(-1\), indicando que aún no se han calculado.\\

\textbf{Paso a Paso de la Ejecución: }

\begin{enumerate}
    \item Llamada inicial: \( DP(\texttt{abc}, \texttt{adc}, 0, 0, \text{Cache}) \).
    \begin{itemize}
        \item \( S_1[0] = \texttt{a}, S_2[0] = \texttt{a} \). Como son iguales, el algoritmo calcula \( \text{remp} = 0 + DP(1, 1) \).
        \item También calcula \( \text{elim} = 1 + DP(1, 0) \) y \( \text{ins} = 1 + DP(0, 1) \), mientras que no aplica transposición porque \( S_1[0] \neq S_2[1] \).
    \end{itemize}

    \item Segunda llamada: \( DP(\texttt{abc}, \texttt{adc}, 1, 1) \).
    \begin{itemize}
        \item \( S_1[1] = \texttt{b}, S_2[1] = \texttt{d} \). Como son diferentes, calcula \( \text{remp} = 1 + DP(2, 2) \).
        \item También calcula \( \text{elim} = 1 + DP(2, 1) \) y \( \text{ins} = 1 + DP(1, 2) \), mientras que no aplica transposición.
    \end{itemize}

    \item Tercera llamada: \( DP(\texttt{abc}, \texttt{adc}, 2, 2) \).
    \begin{itemize}
        \item \( S_1[2] = \texttt{c}, S_2[2] = \texttt{c} \). Como son iguales, calcula \( \text{remp} = 0 + DP(3, 3) \).
        \item También calcula \( \text{elim} = 1 + DP(3, 2) \) y \( \text{ins} = 1 + DP(2, 3) \), mientras que no aplica transposición.
    \end{itemize}

    \item Llamada base: \( DP(\texttt{abc}, \texttt{adc}, 3, 3) \).
    \begin{itemize}
        \item Caso base: retorna 0 porque ambas cadenas han sido procesadas completamente.
    \end{itemize}
\end{enumerate}

\textbf{Matriz de Caché al Final: }

\[
\text{Cache} =
\begin{bmatrix}
0 & 1 & 2 & 3 \\
1 & 1 & 2 & -1 \\
2 & 2 & 1 & -1 \\
3 & -1 & -1 & 0 \\
\end{bmatrix}
\] \\

\textbf{Resultado}

El costo mínimo para transformar \( S_1 \) en \( S_2 \) es \( 1 \). La única operación necesaria es reemplazar \( \texttt{b} \) por \( \texttt{d} \).


\subsubsection{Análisis de complejidad espacial y temporal}

    La implementación de programación dinámica utiliza una tabla de caché para almacenar el costo mínimo de edición para cada par $(i,j)$, permitiendo así que cada estado (eliminación, 
    inserción, sustitución y transposición) realice un número constante de operaciones implicando un costo temporal $O(1)$ ya que solo consideramos la consulta al costo, aunque, en el peor caso, será necesario llenar por completo la tabla de caché, la cual tiene 
    dimensiones $(n+1) \times (m+1)$, donde $n$ y $m$ son los largos de $S1$ y $S2$, respectivamente. Esto significa que hay \(O(n \times m)\) estados posibles, y cada uno se evaluará a lo sumo una vez, 
    por lo que la complejidad temporal es \(O(n \times m)\). En cuanto al análisis espacial, el uso de la tabla de caché la que utiliza mayor espacio, lo que 
    implica que la complejidad espacial también es \(O(n \times m)\).
