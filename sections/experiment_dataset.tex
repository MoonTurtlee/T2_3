\begin{itemize}
    \item \textbf{1. IgualLargo.txt}
    \begin{itemize}
        \item Este archivo contiene pares de cadenas de igual longitud. Cada par está identificado por un número que indica la longitud de las cadenas, seguido de las cadenas mismas. 
        \item La utilizacion de este dataset se utiliza a modo de estudiar como varia el tiempo de ejecucion de ambos algoritmos sobre casos aleatorios.
    \end{itemize}
    \item \textbf{2. LetrasRepetidas.txt}
    \begin{itemize}
        \item Este archivo incluye pares de cadenas donde los caracteres de cada cadena son iguales y repetidos.
        \item Este dataset se utiliza para estudiar el peor caso posible teoricamente, en el cual se pueden realizar todas las operaciones en cada llamada
    \end{itemize}
    \item \textbf{3. Transposiciones.txt}
    \begin{itemize}
        \item Este archivo contiene pares de cadenas que difieren únicamente en una o más transposiciones de caracteres consecutivos.
        \item Este dataset se utiliza a modo de estudiar la complejidad adicional que proporcionan las transposiciones y la optimización del costo.
    \end{itemize}
    \item \textbf{4. VacioS1.txt y VacioS2.txt}
    \begin{itemize}
        \item Estos archivos incluyen cadenas donde una de las cadenas está vacía, mientras que la otra tiene diferentes longitudes.
        \item Permite estudiar casos donde todas las operaciones son solamente inserciones o eliminaciones.
    \end{itemize}
\end{itemize}