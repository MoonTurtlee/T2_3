Los resultados obtenidos destacan la relevancia de seleccionar el enfoque algorítmico adecuado según las características 
del problema planteado. La programación dinámica, al reducir cálculos redundantes mediante el almacenamiento de resultados 
intermedios, demuestra ser eficaz para escenarios con grandes volúmenes de datos o configuraciones complejas. Por otro lado, 
la fuerza bruta, pese a su alta complejidad, se mantiene como una herramienta competitiva en casos simplificados donde las 
restricciones de memoria son críticas. Así tambien discernir la inclusion de transposiciones depende expclusivamente de lo que se busque,
si es prioritario minimizar los costos de edicion son una buena opcion, sin embargo si solo se buscan soluciones factibles su inclusion es
presindible.\\