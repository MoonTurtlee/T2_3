En el ámbito de las Ciencias de la Computación, el análisis y diseño de algoritmos constituye un área fundamental, ya que establece las bases para resolver problemas complejos de manera eficiente. Uno de los problemas clásicos en este campo es la comparación de cadenas, esencial en aplicaciones como la corrección ortográfica, el reconocimiento de patrones y la bioinformática. La distancia de Damerau-Levenshtein, que mide el costo mínimo para transformar una cadena en otra mediante operaciones de inserción, eliminación, reemplazo y transposición, proporciona una métrica versátil y ampliamente utilizada.

A pesar de su relevancia, este problema plantea desafíos significativos en términos de eficiencia computacional, especialmente cuando se utilizan algoritmos de alta complejidad como la fuerza bruta. Las soluciones modernas, como las basadas en programación dinámica, prometen una mejora considerable, aunque con un costo adicional de memoria. Sin embargo, existe una brecha en la comprensión detallada del impacto de las operaciones individuales y los escenarios específicos sobre el rendimiento de estos algoritmos.

El propósito de este informe es estudiar y comparar ambos enfoques, fuerza bruta y programación dinámica, en el contexto del cálculo de la distancia de Damerau-Levenshtein. A través de un diseño detallado, implementaciones prácticas y experimentos controlados, se busca determinar cómo los costos asociados a cada operación afectan la complejidad temporal y espacial de las soluciones. Además, se evalúan casos particulares, como cadenas vacías o con caracteres repetidos, para identificar patrones de rendimiento y posibles áreas de optimización.


