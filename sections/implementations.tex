A continuacion se detallan los archivos utilizados y el contenido de cada uno, además se puede encontrar el link al repositorio
de Github el cual puede ser revisado en el apendice \ref{github}. 

\begin{itemize}
    \item \textbf{Archivos .cpp y .hpp}
    \begin{itemize}
        \item \textbf{DP.cpp:} Algoritmo de programacion dinamica.
        \item \textbf{BF.cpp:} Algoritmo de fuerza bruta.
        \item \textbf{utils.hpp:} Funcion de lectura de archivos para los costos, funciones de costos, funcion de tamaño espacial, funcion de reconstruccion.
    \end{itemize}
    \item \textbf{Generadores}
    \begin{itemize}
        \item \textbf{GeneradorData.py: } Genera los datasets a utilizar.
        \item \textbf{GenerarCostos.py: } Genera los costos de las operaciones de insercion, eliminacion, reemplazo y transposicion.
        \item \textbf{Graficos.py: } Genera los graficos con los resultados almacenados en BF.txt y DP.txt.
    \end{itemize}
    \item \textbf{Archivos .txt}
    \begin{itemize}
        \item \textbf{cost\_operacion.txt: } estos archivos contienen las matrices de costos generadas. 
        \item \textbf{DP.txt y BF.txt: } contienen los datos de salida de cada ejecucion
        \item \textbf{(datasets).txt: } estos archvos contienen los casos de prueba utilizados.
    \end{itemize}
    \item \textbf{Carpetas}
    \begin{itemize}
        \item \textbf{Codigos: } Se tienen los .cpp junto con los costos y los datasets en archivos .txt  
        \item \textbf{Generacion: } Contiene los generadores
    \end{itemize}
\end{itemize}
